\documentclass{article}

\usepackage[nottoc,numbib]{tocbibind}

\newcommand{\mrm}[1]{\mathrm{#1}}

\begin{document}

\begin{center}{\bf \LARGE Implementation}\end{center}

\tableofcontents

\section{Data Structures and Algorithms}

\subsection{Caustics}

I achieved caustics via the photon mapping method described in detail by Jensen
et al. in
\cite{coursenotes}. The method requires a forward ray tracing preprocessing
step during which light-surface interactions are recorded. It also mandates
modifications to lighting computations during the normal rendering step. As
Jensen recommends, I used the k-D tree data structure for performance reasons.
Also for performance reasons, I implemented projection maps for the
preprocessing step.

Code related to the preprocessing step, radiance estimates, and projection maps
is in {\tt photonmap.[ch]pp}. My k-D tree implementation is in {\tt kdtree.hpp}.
Forward ray tracing code is {\tt rt.[ch]pp}. In particular, see the {\tt
RayTracer::raytrace\_russian} method.

\subsubsection{Forward Ray Tracing}

The photon mapping preprocessing step entails shooting a set number of rays
(`photons') from each light in the scene and recording the interactions of the
photons with diffuse surfaces. It is a Monte Carlo algorithm; photons are shot
in random directions, and their subsequent behaviour is randomly decided (in the
Russian Roulette style briefly discussed in class). The intensity of each photon
must be chosen based on several factors, such as the user-specified unit
distance, the number of photons to be shot, and the fullness of the projection
map.

\label{intensity}

Determining the right intensity to assign to each photon was very difficult, and required me to fiddle with parameters and add new ones until it worked well in the general case. I compute the energy of each photon as

$$E = \frac{4 \pi d_u^2 \cdot C_{\mrm{light}}}{f_{\mrm{light}}P}$$

where $d_u$ is the user-specified unit distance (which depends on the scale of the scene); $C_{\mrm{light}}$ is the intensity (or colour) of the light; $f_{\mrm{light}}$ is the $r^2$ falloff coefficient of the light; and $P$ is the user-specified number of photons to be shot. Note that $4\pi d_u^2$ is the area of a unit sphere around the light. Note also that if we use projection maps, then we must reduce the number of photons shot based on the occupied area of the projection map.

Instead of generating multiple recursive rays in the reflective and refractive
directions at every specular event -- as is done in backward ray tracing
-- we choose to have photons perform diffuse or specular interactions -- or,
alternatively, be absorbed -- based on the
attenuation that would be applied to the light in each case. For example, if a
material has diffuse colour $(0.3, 0.3,
0.3)$ and specular colour $(0.1, 0.1, 0.1)$, then

$$P(\mrm{diffuse\; reflection}) = 0.9 / 3 = 0.3$$
$$P(\mrm{specular\; interaction}) = 0.3 / 3 = 0.1$$
$$P(\mrm{absorb}) = 1 - 0.3 - 0.1 = 0.6$$

Note that this imposes the constraint that the specular plus diffuse coefficients
f a material cannot exceed $1$; this is also a true in the real world.

Every time we hit a diffuse surface, we store the photon's intensity, along with
the hit coordinates, in a k-D tree which is used later in the rendering phase.
To simulate caustics, we are only concerned with photons that hit diffuse
surfaces after going through at least one specular interaction; these are the
only photons we store in the map.

\subsubsection{Projection Maps}

During the caustic photon map preprocessing phase, we spend time shooting a
photon in a given direction only after checking that there are specular objects
in that direction by querying the light's projection map.

Projection maps are bit arrays in which individual entries are mapped to areas on
a conceptual sphere wrapped around the light. If an entry is set, then there are
specular objects in that direction. The mapping uses polar coordinates (and so
is non-uniform).

To fill in projection maps, we shoot rays in the directions corresponding
to each entry and see whether they hit specular objects. All entries
neighbouring an entry set in this way are also set to guarantee that we don't
miss anything despite sparse sampling.

Projection maps were one of my most difficult features to implement, as projecting vectors onto a unit sphere and mapping the sphere to an array required several error-prone bi-directional trigonometric conversions. Take a look at my {\tt ProjectionMap} class in {\tt photonmap.cpp} to see what I mean. It was only after implementing a debugging feature to draw the outline of my projection map that I was able to get things right. Due to the non-uniformity of my parametrization of the spherical map, it was also difficult to compute the proportion of the occupied area of the map. This computation is required for proper caustic intensities (see section \ref{intensity}).

Here are the timings of the photon mapping proprocessing phase for a simple
image with three specular objects with and without using projection maps (see
{\tt caustics1.png}). 10\% of the light's spherical projection map is occupied
in this scene.

\begin{center}
\begin{tabular}{|l|r|r|r|} \hline
 System & Without map & With map & Speedup \\\hline
 FX-6100 & 32.5s & 10.1s & 3.22x \\\hline
 Core i5-3317U & 26.4s & 8.3s & 3.18x \\\hline
\end{tabular} \\
\end{center}

As you can see, the use of projection maps can significantly reduce the runtime
of this preprocessing step. Often, though, the main rendering phase accounts for
the brunt of the program's runtime and so this optimization makes little
difference overall. The optimization is most useful when you want high-quality
caustic effects, as these require large numbers of generated photons.

There is no benefit, or even a slowdown, if we use projection maps in sparse
scenes, as it is less useful to avoid the ray intersection code. {\tt
caustics2.png} is such a scene. Here are its timings with and without projection
maps:

\begin{center}
\begin{tabular}{|l|r|r|r|} \hline
 System & Without map & With map & Speedup \\\hline
       FX-6100 & 21.6s & 30.3s & 0.71x \\\hline
 Core i5-3317U & 25.6s & 37.5s & 0.68x \\\hline
\end{tabular} \\
\end{center}

Even for sparse scenes, though, the projection map is beneficial, as it guarantees that specular objects get their fair share of photons.

\subsubsection{Computing Caustic Radiance}

We integrate caustics into our lighting model by simply adding `radiance
estimates' retrieved from our caustic photon map to our other diffuse surface
colour computations (ie, the Phong model).

Given a point, we compute the radiance estimate for that point by querying our
caustic map k-D tree for the $N$ nearest neighbours, where $N$ is a
user-specified parameter. We then return the area-average of the results; that
is, we sum the resulting intensities, and divide by the area
of the smallest disc that includes all of the photons. Given that photon
intensities are initially chosen based on the number of photons to be shot, this
computation yields a reasonable estimate of the true caustic radiance.

\subsubsection{The k-D Tree}

For images with caustics or global illumination, the photon map k-D tree
structure is often the major performance bottleneck (as high quality
effects require that massive number of photons are stored and high numbers of
neighbours are queried).

% Idea: k-D tree/node structure here.

Before the ray tracer's rendering step, it must build the caustic photon k-D
tree from the photons produced by the photon shooting preprocessing step.
This involves repeatedly finding median points on the $x$, $y$, and $z$ axes. My
relatively naive implementation sorts the points to find the median, which
results in an $O(n\lg^2{n})$ algorithm.

Finding the $N$ nearest neighbours of a point in a k-D tree entails a recursive
algorithm that traverses the binary tree, checking at each level whether the
node's point is closer than the current candidates to the target, and deciding
to visit children nodes based on whether the children on that side of the split
can possibly be closer than the current candidates. Usually, we'll need to visit
many leaf nodes before we find the desired number of points. The resulting
algorithm has complexity $O(k\lg{n})$, where $k$ is the number of neighbours to
be retrieved.

\subsection{Constructive Solid Geometry}

Our constructive geometry algorithms iterate through pairs of lists
of line segments (one list for each operand), determining where the next segment
begins or ends and pushing new segments into an output list based on the
operation type (union, difference, or intersection) and the interactions between
the lists. The relevant code is in {\tt csg.[ch]pp}.

The nature of our proposed final scene required us to implement
interfaces between materials other than air for the purpose of refraction (for
example, we needed to be able to accurately model transmission of light from a
glass to a liquid inside the glass and back). We found that the constructive
solid geometry union function provided us with a perfect opportunity to acquire
the necessary information about adjacent mediums.

Thus, our constructive solid geometry union function returns information about a
`from' medium and a `to' medium. In particular, the function records an
interface between two materials when

\begin{itemize}
  \item A line segment for the second operand begins during a line segment for
    the first operand
  \item A line segment for the second operand ends during a line segment for the
    first operand
  \item A line segment for one operand ends within $\epsilon$ units of the
    beginning of a line segment for the other (for some small $\epsilon$)
\end{itemize}

Note that the resulting semantics have the second union operand ``override" the
first in places where they overlap. We did not augment the constructive solid
geometry difference and intersection functions in this way, as coming up with
consistent and intuitive semantics (and writing code for them) was
difficult; moreover, the union functionality was sufficient for our purposes.

Note also that this feature has no effect whatsoever unless the materials
involved are transparent.

To accomodate this feature, our ray tracing code had to be modified to be aware
of the possibility of a `from' medium.

\subsection{Reflection and Refraction}

Reflection and refraction both involve shooting recursive rays in
specific directions. In the case of refraction into a coloured medium, the
light intensity is attenuated as a function of the distance travelled in the
medium.

In our implementation, a single colour dictates the intensities of reflection and
refraction for a material (though either feature can be individually disabled).
The relative proportions of this coefficient that contribute to reflection and
refraction, respectively, are determined by the \emph{Fresnel equations}. In
particular: given an incident angle away from the normal of $\theta_i$,
a transmitted angle away from the normal of $\theta_t$, and refractive indices
$n_1$ and $n_2$ for the incident and transmission materials respectively, the
proportion of specular light which is reflected -- the rest is transmitted -- is
given by

$$\frac{1}{2} \cdot \left(\left(\frac{n_1 \cos\theta_i - n_2 \cos\theta_t}{n_1
\cos\theta_i + n_2\cos\theta_t}\right)^2 + \left(\frac{n_2 \cos\theta_i -
n_1\cos\theta_t}{n_2\cos\theta_i + n_1\cos\theta_t}\right)^2\right)$$

We were motivated to use the Fresnel equations since our important features as
well as our final scene were heavily dependent on refraction, and the Fresnel
equation avoids the discontinuities that arise in simple refraction models
when total internal reflection occurs.

Our primary reference on the Fresnel equations was de Greve \cite{degreve}. The
relevant code is in {\tt rt.cpp}. In particular, see the {\tt fresnel} function.

\subsection{Parallelism}

In assignment 4, I took advantage of the ``embarassing parallelism" of ray
tracing by making my ray tracer multithreaded. In particular, I spawn a number
of threads, each of which repeatedly queries a threadsafe queue data structure.
This structure doles out rows of the image being rendered, and each thread, upon
receiving a row, renders it before asking for another. The result is a
significant speedup.

I have not parallelized the photon mapping preprocessing stage, due to both time
constraints and the fact that the rendering stage is the main bottleneck.

Here are some timings for different systems using different numbers of threads
while rendering my {\tt threading.png} scene:

\begin{center}
\noindent\textbf{Core i5-3317U (two cores hyperthreaded)} \\
\begin{tabular}{|r|r|r|r|} \hline
  \# threads & User time & Clock time & Speedup \\\hline
           6 & 127.1s & 33.1s & 2.51x \\\hline
  4 & 127.0s & 32.8s & 2.53x \\\hline
  2 & 84.3s & 42.4s & 1.96x \\\hline
  1 & 83.3s & 83.1s & 1.00x \\\hline
\end{tabular} \\

\noindent\textbf{FX-6100 (six cores)} \\
  \begin{tabular}{|r|r|r|r|} \hline
  \# threads & User time & Clock time & Speedup \\\hline
           6 & 28.9s & 4.9s & 4.59x \\\hline
           4 & 27.3s & 6.8s & 3.31x \\\hline
           2 & 23.4s & 11.7s & 1.92x \\\hline
           1 & 22.5s & 22.5s & 1.00x \\\hline
  \end{tabular}

\end{center}

As you can see, multithreading my program increased its performance
considerably. Moreover, the improvement is mostly scene-independent.

\section{Methodology and Tools}

\subsection{Language and Style}

While writing my ray tracer, I made extensive use of C++11 features. The program
is compiled with gcc's {\tt -std=c++0x} option. Of particular note is my use of {\tt
std::function} and C++11 lambdas in my ray tracing and primitive ray intersection
functions.

I made use of standard library data structures and algorithms whenever
appropriate. Dynamic memory allocation was never a performance issue.

\subsection{Configurability}

The ray tracer has on the order of $20$ different user-specifiable options that
govern the appearance of generated images. These include things like the sampling
resolutions for various features, the number of photons to shoot for photon mapping,
whether to do special renderings meant to demonstrate certain features, et
cetera.

There are two ways to control every option: you may specify them on the
command line, or you may specify them in a LUA script via the {\tt gr.option}
call. {\tt gr.option} accepts as its argument a string which is interpreted as a
sequence of command-line arguments.

Having {\tt gr.option} is advantageous because it frees the user from having to
remember the desired command-line parameters for a given image. Requiring the
user to pass a command-line string to {\tt gr.option} (rather than having
dedicated LUA functions for every option) gets around the need to
add boilerplate to {\tt scene\_lua.cpp} every time we want to add a new
option.

All of the code related to program options and command-line arguments is in
{\tt cmdopts.cpp}.

\subsection{Performance}

I spent a good deal of time attempting to improve the performance of my ray
tracer in ways unrelated to my project goals. Currently, the biggest performance
bottleneck for caustic-enabled scenes is the k-D tree $N$ nearest neighbours
function, which is usally invoked with values of $N$ on the order of $100$ and
must search through a binary tree containing hundreds of thousands to millions
of photons.

I was able to significantly reduce the time spent in the nearest neighbour
search by adding a maximum distance to the search; beyond this distance, we simply give up
and return whatever we have. This allows us to return early when we search for
photons in dark areas of the scene rather than wasting time gathering photons
which will be too far away to influence the radiance.

Constructive solid geometry objects significantly impacted performance before I
implemented hierarchal bounding boxes (which are now automatically placed around
CSG primitives). Here are some timings with and without these bounding boxes for
my {\tt csgbb1.png} script: \\

\begin{center}
\begin{tabular}{|l|r|r|r|} \hline
 System & No bounding boxes & Bounding boxes & Speedup \\\hline
 FX-6100 & 35.7s & 3.8s & 9.39x \\\hline
 Core i5-3317U & 54.6s & 5.8s & 9.41x \\\hline
\end{tabular} \\
\end{center}

Clearly, this optimization has a significant impact on performance in scenes
with characteristics.

Perhaps unsurprisingly, I doubled the speed of my sphere intersection function
by converting it from a generalized quadric intersection function into one
specific to spheres.

There was one case where initializing a {\tt std::function} with a C++11
lambda caused a dynamic allocation in a hot loop, and initializing
the function outside the loop improved performance by about 10\%. In general,
though, dynamic memory allocation was not a significant contributor to
performance issues, despite my frequent use of standard library containers.

\subsection{Tools}

I made use of the following tools:

\begin{itemize}
  \item I used Valgrind's callgrind tool in order determine the best targets for
optimization (given reasonable k-D tree sizes, the program is CPU-bound).

  \item I used GNU's gdb whenever I needed to debug a crash (although I generally used
print statements for other types of debugging).

  \item I used GIMP, the GNU image manipulation program, to generate several textures
that I used for testing and in demo images.

  \item I used Git for version control. For general interest, I had made $68$ commits
in my project {\tt src} directory at the time of writing.

  \item I used {\tt bitbucket.com} for free private Git hosting.
\end{itemize}

\section{Bugs and Assumptions}

I have run into bugs with CSG objects being clipped. This may be a bug in my CSG hierarchal bounding boxes. The bug rarely surfaces, and so I haven't devoted much of my precious time to fixing it. Nor have I disabled CSG bounding boxes, as they provide an excellent speedup.

\section{Future Possibilities}

Several extensions to photon mapping are described in great detail by Jensen in
\cite{coursenotes}. I am most interested in experimenting with the use of photon
mapping for global illumination, as the images produced through the use of
global illumination techniques can be very impressive. There is also a discussion
of the use of photon mapping to render participating media such as fog; smoke would
be a great addition to my final scene. Subsurface scattering is another possible
extension of photon mapping.

I have already put some effort into achieving global illumination, as there is very little I need to implement in addition to what I already have for caustics. The resulting images are promisig, though not polished yet (see for example {\tt gi1.png} and {\tt gi2.png}).

\section{Acknowledgements}

\begin{itemize}
  \item I received help from Lucie Lemmond with improving my final scene.

  \item One of the textures I used in my final scene was a public domain image I retrieved from the Wikimedia Foundation \cite{wikimedia}.

  \item I'd also like to point out my dependence on the GIMP's excellent features for texture and bump map generation.
\end{itemize}

\section{Code Map}

What follows is a brief description of what you can find in the various source
files, ordered by approximate importance. I have omitted files which were
present in the assignment 4 base code and which haven't been significantly
changed.

\newcommand{\cmditem}[1]{\item {\tt #1}}

\begin{itemize}
    \cmditem{a4.cpp} -- {\tt a4\_render}, the main render function.
    \cmditem{rt.cpp} -- The {\tt RayTracer} class, which encapsulates most of
    the logic for recursive ray tracing (including reflection and refraction).
    \cmditem{primitive.cpp} -- Primitive definitions, including intersection
    testing functions.
    \cmditem{lightingmodel.cpp} -- Lighting computations, including soft shadows
    (in particular, see the {\tt occlusion} function).
    \cmditem{csg.cpp} -- Code for constructive solid geometry.
    \cmditem{photonmap.cpp} -- Photon map creation, radiance estimation, and
    projection map code.
    \cmditem{kdtree.hpp} -- The {\tt kdtree} template class.
    \cmditem{textures.cpp} -- Texture and bump map definitions; support for
    PNG image textures and bump maps.
    \cmditem{texdefs.cpp} -- UV Remapping code and procedural textures and bump
    maps.
    \cmditem{intersection.hpp} -- Some fast intersection functions that are used
    in several places.
    \cmditem{threading.hpp} -- Multithreading support.
    \cmditem{timer.cpp} -- The {\tt ProgressTimer} class, which prints rendering
    time and progress.
    \cmditem{stats.cpp} -- Support for recording and printing statistics.
    \cmditem{cmdopts.cpp} -- Program options and command-line argument parsing.
\end{itemize}

\begin{thebibliography}{9}

\bibitem{coursenotes}
 Henrik Wann Jensen,
 \emph{A practical guide to global illumination using ray tracing and photon mapping}.
 ACM SIGGRAPH 2004 Course Notes, 2004 [ exceptionally detailed, thorough, and readable SIGGRAPH course notes on photon mapping and its extensions ]

\bibitem{degreve}
Bram de Greve, \emph{Reflections and Refractions in Ray Tracing}. November 2006
[ information on using the Fresnel equations for reflection and refraction ]

\bibitem{wikimedia}
  Wikimedia Commons (unknown author), \emph{Graph of the Growth of Boston Schoolchildren in Height and Weight}. 1907 [ public domain image used in my final scene ]

\end{thebibliography}

\end{document}
