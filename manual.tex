\documentclass{article}

\begin{document}

\begin{center}{\bf \LARGE Manual}\end{center}

\tableofcontents

\section{Running the Program}
\label{running_the_program}

The ray tracer is driven by the same style of LUA scene scripts as in assignment
4. It now accepts several additional commands, which are laid out in section
\ref{lua_commands}. The program is run on the command line, and it accepts as
arguments an optional series of command-line flags and name-value pairs (see
section \ref{program_options}) followed by a series of LUA scene scripts to run.
If no scripts are specified, the program attempts to run {\tt scene.lua}. As in
assignment 4, images can be produced from scene files using the {\tt gr.render}
function.

While the program is rendering a scene, it outputs information about elapsed
time and progress.

\section{Using the Features}
\label{using_the_features}

This section describes how to use the features of the ray tracer. Features which
were mandatory components of assignment 4 will not be discussed unless they have
seen changes since assignment 4.

\subsection{Anti-Aliasing}
\label{anti_aliasing}

Adaptive anti-aliasing is used in this ray tracer. It reduces `jaggies' by
performing rendering calculations multiple times for pixels containing colour
discontinuities. There are two parameters to adjust: the sensitivity to
discontinuities (ie, the number of pixels which are selected for anti-aliasing)
and the number of samples used for anti-aliasing a selected pixel. In practise,
anti-aliasing is inexpensive given a reasonable threshold. See also the
{\tt --aa-grid} and {\tt --aa-threshold} program options.

\subsection{Multithreading}
\label{multithreading}

This ray tracer takes advantage of thread-level parallelism to achieve
significant performance boosts. To achieve optimal performance, use the
{\tt --threads} program option to have the program use a number of threads which is
approximately equal to the number of logical cores on the machine.

\subsection{Cone and Cylinder Primitives}
\label{primitives}

It is possible to create cone and cylinder primitives via the {\tt gr.cone} and
{\tt gr.cylinder} commands. These primitives are hierarchal and both have heights
2 and radii 1. The same actions may be performed
upon them as on any other primitive.

\subsection{Texture Mapping}
\label{texture_mapping}

Texture mapping is used in this ray tracer so that users may `map' pictures onto
primitives. In particular, materials with associated textures have their diffuse
colour coefficients determined by the texture.

A user may create a texture using the {\tt gr.texture} function. The function
accepts a single string argument which is either the name of a procedural
texture or the path of a PNG image. The only procedural textures that are
supported are `redcheck', `greencheck', `bluecheck', and `whitecheck', which are
checker patterns of varying colours. After a texture is created, it may be
attached to a material using the {\tt material:set\_texture} method.

Each primitive has a parametrization by which PNG images are mapped to them; see
section \ref{uv_parametrizations}.


\subsection{Bump Mapping}
\label{bump_mapping}

Bump mapping is used to perturb the surface normals of primitives according to a
user-specified map. Doing so can produce impressive images by having the
direction of light bounces give the impression of detailed shapes; but bump
mapping is usually much less difficult than modeling the detailed shapes
themselves, and rendering simple shapes with bump maps is certainly less
expensive than rendering detailed ones.

Users may create bump maps using the {\tt gr.bumpmap} function. The function accepts
a single string argument which is either the name of a procedural bump map or the
path of a PNG image. The only procedural bump map which is supported is the
`ripple' bump map, which simulates the ripple effect of water which has been
disturbed. After a bumpmap is created, it may be attached to a material using
the {\tt material:set\_bumpmap} method.

Image bump maps must be grayscale PNG images. Normal perturbations are drawn
from bump map images via the image gradients; specifically, normals are
perturbed in the direction of positive gradients. Thus: on a rendered primitive,
areas corresponding to darker areas of the bump map will appear more elevated.

Each primitive has a parametrization by which PNG images are mapped to them; see
section \ref{uv_parametrizations}.

\subsection{UV Parametrizations}
\label{uv_parametrizations}

Texture and bump mapping require parametrizations of each primitive so that
two-dimensional images can be mapped to the three-dimensional primitives. For
this purpose, the ray tracer parametrizes the primitives with respect to
rectangular images in the following ways.

For cube primitives, the image is divided into twelve squares in three rows of
four. Each face on the cube is linearly mapped to one of these squares. If we
number the squares from $1$ to $12$ -- top to bottom, left to right -- then
square $6$ is mapped to the top of the cube, and squares $2$, $5$, $7$, $8$, and
$10$ are mapped to the other faces of the cube as if they had been wrapped
around it. The parametrizations of individual faces are consistent with this
wrapping metaphor.

For sphere primitives, an image texture is mapped to the primitive as if it
had been curled into a top- and bottom-less cylinder around the sphere and then
projected onto the sphere from that position.

For cylinder primitives, the lower half of the image is wrapped around the
non-planar face of the primitive. The upper half of the image is divided equally
into two squares on the left and right side. The maximal circles inside these
squares are mapped to the top and bottom of the cylinder, respectively.

For cone primitives, the lower half of the image is wrapped around the
non-planar face of the primitive. The maximal circle inside the upper half of
the image is linearly mapped to the planar face of the cone.

\subsection{Remapping}
\label{remapping}

It often happens that one has an image that they wish to map to only part of a
primitive as a texture or bump map. There is limited support for doing just that
through the {\tt texture:remapt} and {\tt bumpmap:remap} methods, which accept
single strings and reparametrize the images based on the type of remapping that
the string refers to. Examples of supported remap types are:

\newcommand{\cmditem}[1]{\item {\tt #1}}

\begin{itemize}
  \cmditem{cubetop} -- The entire image is mapped to the top of the cube.
  \cmditem{cubefront} -- The entire image is mapped to the front of the cube.
  \cmditem{cubebot} -- The entire image is mapped to the back of the cube.
  \cmditem{cyltop} -- The entire image is mapped to the top of the cylinder.
  \cmditem{conetop} -- The entire image is mapped to the cone's planar surface.
\end{itemize}

\subsection{Constructive Solid Geometry}
\label{csg}

The idea behind constructive solid geometry is to provide only simple
primitives, like the sphere, cube, cone, and cylinder primitives supported by
this ray tracer, and allow users to combine them in order to create more complex
shapes. To this end, constructuve solid geometry provides three basic operations
on hierarchal nodes: union, difference, and intersection. The ray tracer exposes
these operations in the form of the {\tt gr.union}, {\tt gr.difference}, and
{\tt gr.intersection} functions, which all require exactly two node operands and
return CSG primitives.

The union function combines two hierarchal nodes into a single CSG object. This
had no visible effect except insofar as the resulting object can be used in
other CSG functions.

The difference function takes two hierarchal nodes and returns a CSG node which
contains the non-empty volume of the first node which does not intersect with
the second node. In other words, the difference function \emph{subtracts} its
second operand from its first. In the resulting node, there may be newly exposed
surfaces of the first operand which were formerly adjacent to volume which
intersected with the second subtracted operand. These surfaces take on the
properties (ie, the materials and normals) of the intersecting volume of the
subtracted shape.

The intersection function returns a CSG node which contains only the volumes
within which its children intersect. Every surface in the resulting node will
correspond with a surface in one of the children volumes; and it is from these
corresponding surfaces that the properties of the resulting CSG node are derived
(ie, the material and normals).

In this ray tracer, CSG union nodes have the special property that they yield
information about interfaces between component volumes. See section
\ref{csg_union}.

\subsection{Reflection, Refraction and Transparency}
\label{rrt}

The reflectivity, refraction and transparency of materials are determined only by
their specular colour coefficients and refractive indices.

\begin{itemize}
  \item Any material with null specular coefficients is opaque and non-reflective.
  \item Any material with nonzero specular coefficients and a refractive index
    which has not yet been set is opaque and reflective. The intensity and
    colour of reflected light depends only on the incoming light and the
    speciular coefficients of the material.
  \item Any material with nonzero specular coefficients and a refractive index
    which has been set is transparent (and necessarily reflective as well). The
    intensity and colour of transmitted and reflected light is determined by the
    incoming light, the specular coefficients, and the Fresnel equations.
\end{itemize}

\subsection{CSG Union and Medium Interfaces}
\label{csg_union}

A problem with modeling refraction in ray tracers is that there is no obvious
way to support refraction between two non-air mediums. For example: to model
liquid in a glass, you might think to simply place a cylinder of liquid directly
adjacent to the sides of the glass object. This will not work, though, because
ray tracers typically ignore collisions that occur within a certain distance of
the source, and because there is no way for the program to know whether there is
meant to be air between the two objects.

The problem is solved in this ray tracer via the CSG union function. Namely, CSG
union nodes yield information about interfaces between their component volumes.
If you wish to create an object (such as a glass of liquid) which contains
interfaces between non-air mediums, you must do so using the union function.

The exact semantics of this aspect of the union function are a bit complex, but
an intuitive understanding can be achieved via the following principles:

\begin{itemize}
  \item Whenever the volume of the left union operand ends within $\epsilon$
    units of where the right operand's volume begins (where $\epsilon$ is some
    hard-coded constant), that is taken to be an interface between the volumes.
  \item Whenever the two operands intersect, the second operand takes priority
    -- ie, the second operand determines the volume's material.
  \item Whenever the volume of the second operand begins inside the volume of
    the first operand, that is (obviously) an interface between the two volumes.
\end{itemize}

\subsection{Soft Shadows}
\label{soft_shadows}

The naive approach to shadows is to use point lights and to colour pixels one of
two colours based on whether they are occluded. This is often undesirable, as in
the real world shadows give way gradually to light; the edges of shadows are
blurry. This ray tracer supports these ``soft shadows".

To enable soft shadows, you must simply make your light into a spherical area
light using the {\tt light:set\_radius} method.

Soft shadows are achieved via repeated sampling. Sample density, and thus shadow
quality, can be managed through the {\tt --shadow-grid} program option.

\subsection{Caustics}
\label{caustics}

The backward ray tracing employed by typical distributed ray tracers cannot
easily simulate a plethora of caustic effects that light often produces in the
real world. In order to achieve these effects, this ray tracer uses a method
called `photon mapping'. Photon mapping employs a preprocessing phase which
involves repeatedly tracing randomly directed photons from each light through
the scene.

Photon mapping is disabled by default. If you wish to produce caustic effects,
you must provide a nonzero integer argument to the {\tt --caustic-num-photons}
program option, which determines how many photons are shot during the
preprocessing step (a reasonable value for typical scenes is $1000000$). Also
important is the {\tt --caustic-num-neighbours} program option, which determines
how many photons are used in each colour estimate (default: $100$).

In general, the quality of caustic effects will increase as both parameters
increase (though the {\tt --caustic-num-neighbours} parameter should always be a
small fraction of the other). Small values will result in undesirable noise.

You should also be aware that photon mapping is extremely sensitive to lighting
parameters. The most relevant ones are the third falloff parameter for lights,
which must not be zero, and the unit distance (see the {\tt --unit-distance} program
option). If photon mapping fails to produce reasonable results, check these
parameters.

\section{LUA Commands}
\label{lua_commands}

The following is a listing of the LUA commands and methods supported by the ray
tracer. For the sake of conciseness and clarity, commands which were mandatory
components of assignment 4 will be ommitted (though they are also available).


\begin{itemize}

  \cmditem{gr.cone(name)} -- Return an hierarchal cone with the specified name.

  \cmditem{gr.cylinder(name)} -- Return an hierarchal cylinder with the
  specified name.

  \cmditem{gr.texture(str)} -- Return the texture object described by {\tt str}.
  {\tt str} may be either the relative path of an RBG PNG image file or one of the
  strings {\tt redcheck}, {\tt greencheck}, {\tt whitecheck}, {\tt bluecheck}. See section
  \ref{bump_mapping}.

  \cmditem{gr.bumpmap(str)} -- Return the bumpmap object described by {\tt str}.
  {\tt str} may be either the relative path of a grayscale PNG image file or the
  string {\tt sinewaves}. See section \ref{texture_mapping}.

  \cmditem{texture:remapt(str)} -- Remap the specified texture as described by
  {\tt str}, which must be one of the strings {\tt cubetop}, {\tt cubefront}, {\tt cubebot},
  {\tt cyltop}, {\tt conetop}. See section \ref{texture_mapping}.

  \cmditem{bumpmap:remap(str)} -- Remap the specified bumpmap as described by
  {\tt str}, which must be one of the strings {\tt cubetop}, {\tt cubefront},
  {\tt cubebot},
  {\tt cyltop}, {\tt conetop}. See section \ref{bump_mapping}.

  \cmditem{material:set\_texture(texture)} -- Have the material use the given
  texture object for its diffuse coefficients.

  \cmditem{material:set\_bumpmap(bumpmap)} -- Have the material use the given
  bump map to perturb normal vectors.

  \cmditem{material:set\_ri(ri)} -- Sets the refractive index of the object to
  {\tt ri}. See section \ref{rrt}.

  \cmditem{gr.union(c1, c2)} -- Creates a constructive solid geometry node representing
  the union of {\tt c1} and {\tt c2}. See section \ref{csg}.

  \cmditem{gr.intersection(c1, c2)} -- Creates a constructuve solid geometry node
  representing the intersection of {\tt c1} and {\tt c2}. See section
  \ref{csg}.

  \cmditem{gr.difference(c1, c2)} -- Creates a constructive solid geometry node
  representing the difference between {\tt c1} and {\tt c2} (in other words, the
  resulting node is $c1 \setminus c2$). See section
  \ref{csg}.

  \cmditem{light:set\_radius(rad)} -- Sets the radius of the light to {\tt rad} (see
  section \ref{soft_shadows}.

  \cmditem{gr.option(str)} -- Pass an option string to the program. Option strings
  take the same form as command-line arguments. See section
  \ref{program_options}.

  \cmditem{gr.set\_miss\_colour(colour)} -- Sets the colour to be used when a
  ray fails to hit any objects in a scene.

\end{itemize}

\section{Program Options}
\label{program_options}

Program options are global values which affect the appearance of the image
produced as well as the runtime of the program.

There are two types of options: flag (boolean) options, which are by default
disabled and can only be enabled, and single-value options, which must always be
accompanied by either integer or real values.

There are two ways to toggle flags or specify single-value options: via the
command line and via the {\tt gr.option} LUA command. On the command line, any
number of strings (for flag options) or whitespace-separated string-value pairs
(for single-value options) can be specified \emph{before} the first scene file.
The {\tt gr.option} command accepts a single string which has the same form as a
sequence of command-line arguments (note in particular that you can modify many
program options through a single call to {\tt gr.option}).

The following is a comprehensive listing of valid program options, in
approximately descending order of usefulness.

\begin{itemize}

  \cmditem{--threads N} -- Have the program use up to N threads. N should be
  the number of logical cores on the machine for maximum performance. Default:
  6.

  \cmditem{--aa-grid N} -- Specify that the program should use an NxN grid when
  it performs anti-aliasing for a single pixel. Higher values may improve image
  quality but will reduce performance. Default: 3.

  \cmditem{--aa-threshold R} -- Specify the discontinuity threshold above which
  anti-aliasing should be performed on a pixel (see section
  \ref{anti_aliasing}). Higher values may reduce image quality while improving
  performance. Default: $0.5$.

  \cmditem{--caustic-num-photons N} -- Specify the number of caustic photons to be
  shot during the caustic photon mapping preprocessing stage (see section
  \ref{caustics}). Higher values will reduce performance while improving caustic
quality. See also {\tt --caustic-num-neighbours}. Default: 0.

  \cmditem{--caustic-num-neighbours N} -- Specify the number of neighbours to be
  used in caustic radiance estimations (see section \ref{caustics}). Higher
  values will reduce performance while improving caustic quality (though this is
  dependent on the value of --caustic-num-photnos). Default: 100.

  \cmditem{--shadow-grid N} -- Specify that the program sould use an NxN grid when
  it computes soft shadows. Higher values may improve shadow quality but may
  reduce performance. Default: 4.

  \cmditem{--energy-fudge D} -- Have the photon map adjust the amount of energy
  used in photon shooting for every light proportionally to D. This is useful
  for situations when you want to abandon a physically realistic model for
  artistic reasons. In particular, you may wish to have objects produce caustic
  effects despite being far away from the relevant light. Default: 1.

  \cmditem{--unit-distance D} -- Specify the unit distance. Higher unit distances
  will increase the strength of lights with non-zero second and third falloff
  coefficients and will increase the brightness of caustics. This option is
  useful when dealing with scripts with very large or very small scenes (ie,
  scripts which scale their objects by very large or very small amounts).
  Default: 1.

  \cmditem{-t, --timing} -- Print the runtimes of various phases of the program
  after it's done rendering a scene.

  \cmditem{-s, --silent} -- Don't print normal output. Will still print errors.

  \cmditem{-S, --stats} -- Print various statistics after the program is done
  rendering a scene.

  \cmditem{-v, --verbose} -- Print debug output.

  \cmditem{-d, --debug} -- Print debug output and suppress normal output.

  \cmditem{--disable-aa} -- Disables anti-aliasing. Equivalent to `--aa-grid 1'.

  \cmditem{--disable-bv} -- Disables mesh bounding volumes. Will likely reduce
  performance.

  \cmditem{--draw-aa} -- Colours pixels selected for anti-aliasing red.

  \cmditem{--draw-caustic-map} -- Draws caustics photons on the image rather
  than performing normal lighting calculations.

  \cmditem{--draw-caustic-prm} -- Draws red pixels on diffuse surfaces inside
  the range of the caustic projection map for just one light.

  \cmditem{--minres} -- Override scene scripts and always render images with
  resolution 100 by 100.

  \cmditem{--lores} -- Override scene scripts and always render images with
  resolution 256 by 256.

  \cmditem{--midres} -- Override scene scripts and always render images with
  resolution 512 by 512.

  \cmditem{--hires} -- Override scene scripts and always render images with
  resolution 1024 by 1024.

  \cmditem{--rays} -- Draw a sunny background where rays fail to intersect with
  objects in the scene.

\end{itemize}

\end{document}
